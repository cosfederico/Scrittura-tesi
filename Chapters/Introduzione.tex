\prefacesection{Introduzione}

Per chiunque abbia visitato qualunque sito di divulgazione o social media negli ultimi due anni, sarà stato impossibile non imbattersi in contenuti, fotografici o video, generati dall'Intelligenza Artificiale (IA). L'argomento è vasto, così come la quantità di contenuti diversi generabili tramite IA. Noi ci concentreremo su video basati su "avatar".
% Tra queste tecnologie, si identificano sistemi di IA in grado di realizzare video in movimento raffiguranti persone reali. È possibile, a partire da una fotografia o un breve video di una persona reale, e un testo di riferimento, generare un video della persona raffigurata che parla esponendo ad alta voce il testo indicato.

Un video basato su avatar è un video parlato rappresentante una persona, il cui movimento della bocca, della faccia, e talvolta anche del corpo, è stato generato tramite IA. Spesso gli avatar sono persone reali, le quali hanno messo a disposizione la loro figura affinché potesse essere animata. In altri casi, un avatar può anche essere creato a partire da immagini di persone non reali, generate a loro volta attraverso strumenti di IA generativa (es. Stable Diffusion, Midjourney, DALL-E). Si crea di fatto una copia digitale della persona raffigurata, con il potere di fargli dire qualsiasi cosa. Vi è un testo di riferimento, e a partire da una fotografia o un breve video del soggetto, viene generato un video rappresentante il soggetto che espone ad alta voce il testo indicato.\footnote{Naturalmente la realizzazione di questi video richiede anche la presenza di una voce parlata. Per questo si utilizzano modelli generativi di tipo TextToSpeech, ma la voce utilizzata tipicamente non è la voce della persona rappresentata, bensì una voce di servizio. Nel nostro studio sono stati utilizzati modelli di voce di servizio.}

Si identificano quindi due nuove categorie di video, i video reali, raffiguranti una persona reale registrati fisicamente, e i video "fake", raffiguranti a loro volta una persona reale, ma la cui voce e il cui movimento del corpo sono stati realizzati tramite generatori IA. Questa tecnologia ha trovato grande fortuna nel mondo della pubblicità e della istruzione, dove sempre maggiori sono i costi per la registrazione di video in presa diretta. L'utilizzo di tali sistemi permette di realizzare video senza doverli registrare fisicamente, riducendo di molto i costi di produzione. L'obiettivo di questo studio è valutare se tali video fake possono avere la stessa efficacia comunicativa di un video reale, oppure se la natura artificiale di tali video può risultare un ostacolo abbastanza grande nella comprensione e fruizione del contenuto video tale da poter essere misurato attraverso un esperimento scientifico, rendendo questa sorgente tecnologia non ancora pronta, o inadatta. 

\section*{Struttura dell'esperimento}

L'esperimento è stato strutturato come segue:
\begin{enumerate}
\item Identificazione di brevi video educativi reali (max. 5 minuti)
\item Generazione di doppioni fake a partire dai video reali identificati, dove il contenuto informativo è lo stesso, ma il video, così come la voce del soggetto, sono stati generati tramite IA
\item Presentazione di un esperimento fittizio, con proposta in double blind\footnote{Vengono proposti due video a un soggetto volontario, ma quest'ultimo non sa che uno dei due video è fake} di due video a soggetti volontari, i quali sono ignari della vera natura dell'esperimento e in particolare della natura dei video\footnote{A ogni soggetto viene sempre mostrato un video reale e un video fake.}
\item Sessione di domande a risposta aperta sui contenuti discussi nei video, per valutare il grado di comprensione dei contenuti proposti
\item Acquisizione di dati multimodali durante l'esperimento e in particolare durante la visione dei video, tra cui espressioni del viso, battito cardiaco, ritmo respiratorio e tracciamento degli occhi.
\item Analisi ed elaborazione dei dati acquisiti, alla ricerca di cluster indicativi di un fattore di rilevanza/non rilevanza della natura dei video per la comprensione e la fruizione del video visionato
\end{enumerate}

Inizieremo con una breve spiegazione di come questi tipi di video vengono generati, secondo la letteratura attuale, per poi entrare nel dettaglio della ricerca condotta. Infine, verranno tratte le dovute conclusioni in base a quanto trovato.

\clearpage