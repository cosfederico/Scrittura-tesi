\prefacesection{Introduzione}

Per chiunque abbia visitato qualunque sito di divulgazione o social media negli ultimi due anni, sarà stato impossibile non imbattersi in contenuti, fotografici o video, generati dall'Intelligenza Artificiale (IA). Tra queste tecnologie, si identificano sistemi in grado di clonare l'aspetto di una persona reale, permettendo di far dire a questa persona qualsiasi cosa, partendo semplicemente da un loro video o una loro fotografia, e un testo di riferimento. Tali video prendono il nome di video basati su "avatar".

Più tecnicamente, un video basato su avatar è un video parlato in cui il movimento della bocca, della faccia, e talvolta anche del corpo, così come spesso anche il suono della voce, della persona rappresentata è stato generato tramite IA. Spesso gli avatar sono persone reali, le quali hanno messo a disposizione la loro figura affinché potesse essere animata. In altri casi, un avatar può anche essere creato a partire da immagini di persone non reali, generate a loro volta attraverso strumenti di IA generativa (es. Stable Diffusion, Midjourney, DALL-E). Si crea di fatto una copia digitale della persona raffigurata, con il potere di fargli dire qualsiasi cosa. Vi è un testo di riferimento, e a partire da una fotografia o un breve video del soggetto, viene generato un video rappresentante il soggetto che espone ad alta voce il testo indicato.\footnote{Naturalmente la realizzazione di questi video richiede anche la presenza di una voce parlata. Per questo si utilizzano modelli generativi di tipo TextToSpeech, i quali a partire da un testo generano il suono di una voce. La voce generata può essere la voce stessa della persona raffigurata o anche una voce di servizio.}

Distinguiamo quindi due tipi diversi di video, i video reali, raffiguranti una persona reale registrati fisicamente, e i video "fake", raffiguranti a loro volta una persona reale, ma la cui voce e il cui movimento del corpo sono stati realizzati tramite generatori IA. Questa tecnologia ha trovato grande fortuna nel mondo della pubblicità e della istruzione, dove sempre maggiori sono i costi per la registrazione di video in presa diretta. L'utilizzo di tali sistemi permette di realizzare video senza doverli registrare fisicamente, riducendo di molto i costi di produzione. L'obiettivo di questo studio è valutare se tali video fake possono avere la stessa efficacia comunicativa di un video reale, oppure se la natura artificiale di tali video può risultare un ostacolo abbastanza grande nella comprensione e fruizione del contenuto video tale da poter essere misurato attraverso un esperimento scientifico, rendendo questa sorgente tecnologia non ancora pronta per l'integrazione nel mondo dell'educazione multimediale, o inadatta. 

\section*{Struttura dell'esperimento}

L'esperimento è stato strutturato come segue:
\begin{enumerate}
\item Acquisizione di brevi video educativi reali (max. 5/10 minuti)
\item Generazione di doppioni fake a partire dai video reali acquisiti, dove il contenuto informativo è lo stesso, ma il video, così come la voce del soggetto, sono stati generati tramite IA
\item Presentazione di un esperimento fittizio, dove un campione di volontari visiona due video, valutando la loro esperienza di visione, ma, a insaputa dei partecipanti, uno dei due video mostrati è stato generato tramite IA
\item Sessione di domande di valutazione dell'esperienza su una scala da 1 a 5, e sessione di domande di comprensione a risposta multipla sui contenuti discussi nei video, per valutare il grado di comprensione dei contenuti proposti
\item Acquisizione di dati multimodali durante l'esperimento e in particolare durante la visione dei video, tra cui le espressioni del viso, il battito cardiaco, il livello di sudorazione, e il movimento degli occhi sullo schermo.
\item Analisi ed elaborazione dei dati acquisiti, alla ricerca di cluster indicativi di un fattore di rilevanza/non rilevanza della natura dei video per la comprensione e la fruizione del video visionato
\end{enumerate}

Inizieremo con una breve spiegazione semplificata di come questi tipi di video vengono generati, secondo la letteratura attuale, per poi entrare nel dettaglio della ricerca condotta. Infine, verranno tratte le dovute conclusioni in base a quanto trovato.

\clearpage